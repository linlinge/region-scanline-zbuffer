\documentclass[10pt]{article}
\usepackage[UTF8]{ctex}
\usepackage{picinpar,graphicx,bm}
\usepackage{booktabs}
\usepackage{diagbox}
\usepackage{float}


\usepackage{listings}
\usepackage{xcolor}
% 定义可能使用到的颜色
\definecolor{CPPLight}  {HTML} {686868}
\definecolor{CPPSteel}  {HTML} {888888}
\definecolor{CPPDark}   {HTML} {262626}
\definecolor{CPPBlue}   {HTML} {4172A3}
\definecolor{CPPGreen}  {HTML} {487818}
\definecolor{CPPBrown}  {HTML} {A07040}
\definecolor{CPPRed}    {HTML} {AD4D3A}
\definecolor{CPPViolet} {HTML} {7040A0}
\definecolor{CPPGray}  {HTML} {B8B8B8}
\lstset{
    columns=fixed,    
   % numbers=left,                                        % 在左侧显示行号
    frame=none,                                          % 不显示背景边框
    backgroundcolor=\color[RGB]{245,245,244},            % 设定背景颜色
    keywordstyle=\color[RGB]{40,40,255},                 % 设定关键字颜色
    numberstyle=\footnotesize\color{darkgray},           % 设定行号格式
    commentstyle=\it\color[RGB]{0,96,96},                % 设置代码注释的格式
    stringstyle=\rmfamily\slshape\color[RGB]{128,0,0},   % 设置字符串格式
    showstringspaces=false,                              % 不显示字符串中的空格
    language=c++,                                        % 设置语言
    morekeywords={alignas,continute,friend,register,true,alignof,decltype,goto,
    reinterpret_cast,try,asm,defult,if,return,typedef,auto,delete,inline,short,
    typeid,bool,do,int,signed,typename,break,double,long,sizeof,union,case,
    dynamic_cast,mutable,static,unsigned,catch,else,namespace,static_assert,using,
    char,enum,new,static_cast,virtual,char16_t,char32_t,explict,noexcept,struct,
    void,export,nullptr,switch,volatile,class,extern,operator,template,wchar_t,
    const,false,private,this,while,constexpr,float,protected,thread_local,
    const_cast,for,public,throw,std,size_t,__global__,__device__,__host__},
    emph={map,set,multimap,multiset,unordered_map,unordered_set,
    unordered_multiset,unordered_multimap,vector,string,list,deque,
    array,stack,forwared_list,iostream,memory,shared_ptr,unique_ptr,
    random,bitset,ostream,istream,cout,cin,endl,move,default_random_engine,
    uniform_int_distribution,iterator,algorithm,functional,bing,numeric,},
    emphstyle=\color{CPPViolet}, 
    frame=shadowbox,
    basicstyle=\footnotesize\ttfamily,
    tabsize=4,
}


%layout
\usepackage{calc} 
\setlength\textwidth{7in} 
\setlength\textheight{9in} 
\setlength\oddsidemargin{(\paperwidth-\textwidth)/2 - 1in}
\setlength\topmargin{(\paperheight-\textheight -\headheight-\headsep-\footskip)/2 - 1.5in}


\title{计算机图形学 \hspace{2pt}—\hspace{2pt} \begin{large}扫描线Z-Buffer算法 \end{large} }
\author{葛林林}
\begin{document}
\maketitle


\section{预备知识}
\subsection{$obj$文件}
\bm{顶点的表示:}顶点以$v$开头后面跟着该顶点的$x,y,z$三轴坐标,示例如下
$$e.g. \hspace{15pt} v\hspace{3pt} -57.408021\hspace{3pt}196.143694\hspace{3pt}2.816352$$

\bm{纹理坐标的表示:}纹理坐标以$vt$开头。
$$format.\hspace{15pt} vt \hspace{5pt}tu \hspace{5pt} tv$$

\bm{法向量的表示:}法向量的表示以$vn$开头。
$$format.\hspace{15pt} vn \hspace{5pt} nx\hspace{5pt} ny \hspace{5pt} nz$$

\bm{面的表示:}面以$f$开头后面分别跟着面的顶点、纹理坐标、法向量的索引,后面包含三组索引构成一个面,如下所示
$$format.\hspace{15pt} f \hspace{5pt} Vertex_1/Texture_1/Normal_1 \hspace{5pt} Vertex_2/Texture_2/Normal_2 \hspace{5pt} Vertex_3/Texture_3/Normal_3$$
\subsection{OpenGL工程的搭建}


\section{数据结构}
\begin{itemize}
\item{活化多边形表} \\
根据$y_{max}$的值对多边形进行分类:
\begin{itemize}
\item{$a,b,c,d:$多边形所在平面的方程系数}
\item{$id:$多边形的编号}
\item{$dy:$多边形跨越的{\color{red} 剩余}扫描线数目}
\item{$color:$多边形的颜色}
\end{itemize}
\begin{figure}[H]
\begin{center}
\includegraphics[scale=0.6]{structure1.png}
\end{center}
\caption{分类多边形表}
\end{figure}

\item{活化边表}\\
根据$y_{max}$将边进行分类:
\begin{itemize}
\item{$x_l:$左交点的$x$坐标。}
\item{$dx_l:$左交点边上两相邻两条扫描线交点的$x$坐标差。}
\item{$dy_l:$以和左交点所在边相交的扫描线数为初值,以后向下没处理一条扫描线减一。}
\item{$x_r,dx_r,dy_r:$右边的交点的三个对应分量。}
\item{$id:$边所属多边形的编号。}
\item{$dz_x:$沿扫描线向右一个像素,多边形所在平面的深度增量。$dz_x=-\frac{a}{c}(c \neq 0)$}
\item{$dz_y:$沿$y$方向向下移动一根扫描线时,多边形所在平面的深度增量$dz_y=\frac{b}{c}(c \neq 0)$。}
\item{$id:$交点所在多边形的编号。}
\end{itemize}
\begin{figure}[H]
\begin{center}
\includegraphics[scale=0.5]{structure2.png}
\end{center}
\caption{分类边表}
\end{figure}
\end{itemize}

\section{算法}
加载文件$\to$获取深度值$\to$消隐算法

\section{实验结果}
本次实验的

$$a=(y_1*z_2 - y_2*z_1 - y_1*z_3+ y_3*z_1+ y_2*z_3- y_3*z_2)$$
$$b=y*(- x_1*z_2 + x_2*z_1+ x_1*z_3- x_3*z_1- x_2*z_3+ x_3*z_2)$$
$$c=(x_1*y_2  - x_2*y_1- x_1*y_3 + x_3*y_1+ x_2*y_3- x_3*y_2)$$
$$d=- x_1*y_2*z_3 + x_1*y_3*z_2 + x_2*y_1*z_3 - x_2*y_3*z_1 - x_3*y_1*z_2 + x_3*y_2*z_1$$


\section{in和out的讨论}
如下图所示是线段$P_1P_2$为in状态的情况,假设$P_1,P_2,P_3$点对应的坐标分别为$(x_1,y_1),(x_2,y_2),(x_3,y_3)$。则线段$P_1P_2$的斜率为$$k=\frac{y_1-y_2}{x_1-x_2}$$
而线段$P_1P_2$对应的直线方程为:
$$\frac{y-y_1}{y_2-y_1}=\frac{x-x_1}{x_2-x_1}$$
令
$$f(x,y)=\frac{y-y_1}{y_2-y_1}-\frac{x-x_1}{x_2-x_1}$$
则当满足如下公式时则为in状态
$$kf(x_3,y_3)<0$$
既
$$(x_2-x_1)(y_3-y_1)-(y_1-y_2)(x_3-x_1)<0$$
\begin{figure}[H]
\begin{center}
\begin{minipage}[t]{0.45\linewidth}
\includegraphics[scale=0.7]{check_in_or_out1.png}
\caption{$f(x_3,y_3)>0,k<0$}
\end{minipage}
\begin{minipage}[t]{0.45\linewidth}
\includegraphics[scale=0.7]{check_in_or_out2.png}
\caption{$f(x_3,y_3)<0,k>0$}
\end{minipage}
\end{center}

\end{figure}

\end{document}